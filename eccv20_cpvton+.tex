% updated April 2002 by Antje Endemann
% Based on CVPR 07 and LNCS, with modifications by DAF, AZ and elle, 2008 and AA, 2010, and CC, 2011; TT, 2014; AAS, 2016; AAS, 2020

\documentclass[runningheads]{llncs}
\usepackage{graphicx}
\usepackage{comment}
\usepackage{amsmath,amssymb} % define this before the line numbering.
\usepackage{color}

% INITIAL SUBMISSION - The following two lines are NOT commented
% CAMERA READY - Comment OUT the following two lines
\usepackage{ruler}
\usepackage[width=122mm,left=12mm,paperwidth=146mm,height=193mm,top=12mm,paperheight=217mm]{geometry}


\begin{document}
% \renewcommand\thelinenumber{\color[rgb]{0.2,0.5,0.8}\normalfont\sffamily\scriptsize\arabic{linenumber}\color[rgb]{0,0,0}}
% \renewcommand\makeLineNumber {\hss\thelinenumber\ \hspace{6mm} \rlap{\hskip\textwidth\ \hspace{6.5mm}\thelinenumber}}
% \linenumbers
\pagestyle{headings}
\mainmatter

\def\ECCVSubNumber{3284}  % Insert your submission number here

\title{ CP-VTON+: Cloth Shape and Texture Preserving Image-Based Virtual Try-On 
%Image-based Virtual Try-On: its limitations and an improvement
} % Replace with your title

% INITIAL SUBMISSION 
%\begin{comment}
\titlerunning{ECCV-20 submission ID \ECCVSubNumber} 
\authorrunning{ECCV-20 submission ID \ECCVSubNumber} 
\author{Anonymous ECCV submission}
\institute{Paper ID \ECCVSubNumber}
%\end{comment}
%******************

% CAMERA READY SUBMISSION
\begin{comment}
\titlerunning{CP-VTON+}
% If the paper title is too long for the running head, you can set
% an abbreviated paper title here
%
\author{Matiur Rahman Minar\inst{1}\orcidID{0000-0002-3128-2915} \and
Thai Thanh Tuan\inst{2,3}\orcidID{1111-2222-3333-4444} \and
Heejune Ahn\inst{3}\orcidID{2222--3333-4444-5555}  \and
Paul Rosin\inst{3}\orcidID{2222--3333-4444-5555}   \and
Yukun Lai\inst{3}\orcidID{2222--3333-4444-5555}  }
%
\authorrunning{Matiur Rahman Minar et al.}
% First names are abbreviated in the running head.
% If there are more than two authors, 'et al.' is used.
%
\institute{Seoul National University of Science and Technology, Seoul  08544, South Korea \and
Cardiff University, Cardiff, 69121 Heidelberg, UK
\email{heejune@seoultech.ac.kr}\\
\url{http://www.springer.com/gp/computer-science/lncs}}
\end{comment}
%******************
\maketitle

\begin{abstract}

 Recently, image-based virtual try-on (VTON) technology has
drawn increasing research attraction for practical online apparel shopping. We reconsider the commonly-used setting: the input of a a pair of try-on cloth and target human image, and 2-step processing methods: first, it warps the try-on cloth to
align with the target human, and then blends the warped cloth with the
target human image. 
The contributions of this work are three-fold. First,
despite the successful demonstration of the previous work, the input data
condition for successful performance has not yet well investigated. Thus, we conduct a performance study with classified inputs in cloth style, and human pose and shape. reveals that the current image-based VTON algorithms have a limited working condition. Second, through our detailed examination of the classified experiment, we identify the problem in the dataset such as improper human segmentation labeling, improper matching with user demands, such as not retaining non-targeted cloth, and weakness of the state-of-the-art algorithms in cloth warping and blending. Finally, we proposed a new image-based algorithm, named CP-VTON+, tackling the observed issues. The experiment on a commonly-used dataset shows CP-VTON+ outperforms the state-of-the-art methods quantitatively: in Intersection-Over-Union and Structural Similarity Index for the same cloth re-try-on, and Inception Score and visual observation for new cloth try-on.     

\keywords{Virtual Try-On, Image-based, Deep-Learning, Quality Comparison}
\end{abstract}




\section{Introduction}

Online fashion market has been growing rapidly every year. Clothing purchasing decisions are very difficult to make with current non-customized information, like the cloth and models' try fit images. Unlike other products, such as electronic devices, whose function, performance, and styles can be expected through few images and specification tables. Fashion apparels have infinite variations in style, forms, colors, texture, and materials.  Also the difference between personal preferences is huge. Therefore, virtual try-on (VTON) is a highly demanding technology for the on-line shopping \cite{zhang2019role}. 

The early VTON technologies were based on 3D computer graphics technology that uses 3D models of target humans and clothing. The 3D models are usually expensive and difficult to obtain. Therefore, recently 2D image-based VTON technologies are being studied in academia and industry, powered by the recent advances in computer vision technologies based on deep learning. 

There have been many studies with different problem settings related to image-based VTON, from clothed human pose transferring using conditional GAN \cite{ma2017pose}, swapping two humans clothes \cite{jetchev2017conditional}, to VTONs with a try-on cloth and a target human image \cite{Han2017VITONAI}. The last configuration with a try-on cloth and a target human image has been considered practical in many papers, \cite{Han2017VITONAI,Wang2018TowardCI}, and more recently \cite{Sun2019ImageBasedVT,Yu_2019_ICCV,jae2019viton}. 
In this paper, we also consider the VTON problem that use the try-on cloth and human images and generated a new virtual image that the target human replaced the current top or bottom cloth with the try-on cloth. Our implementation is also limited to top clothes due to the restricted dataset but the bottom clothes, e.g. pants or skirts would be easier than top cloth cases because they are simpler than upper clothes in style and shapes.

Although the previous image-based VTON studies shows high quality VTON results, our classified analysis on the cloth styles and human posture in the Section 2 reveals significant problems in the previous works. In this paper, we started with evaluating the quality of SCM\cite{BelongieMP02} based-VTON, VITON \cite{Han2017VITONAI}, and CP-VTON \cite{Wang2018TowardCI}, using a cloth and a human image, according to the posture and body type of the human, the degree of occlusion of the clothes, and the texture of clothes with IoU of warped clothes and SSIM of final re-try-on image result. The more recent works in 2019 could not included this work, the general pattern of image based algorithm would be similar to the three algorithms. Even though CP-VTON generates the best quality outputs among them, all algorithms shows similar problems, the mistakes in human representation, improper network cost function, and the inherent limitations of 2D-based approaches.   
%
We would emphasize here that one reason of the seemingly high quality in the existing algorithms are mainly due to the dataset with low-complexity bias, i.e., most clothes are short-sleeved  and monochromatic, and the poses of humans are mostly in an up-right position. Specifically, as it will be shown in the following section, the results with the long-sleeved cloth arm and body posed shows very low quality. In Section 3, we point out 5 serious issues in previous works including CP-VTON \cite{Wang2018TowardCI}.  


\begin{figure}
\centering
\includegraphics[height=8.5cm, scale=1]{figures/cpvton_cpvton+keyresult.png}   % TODO
\caption{The proposed VTON results}
\label{fig:cpvton_cpvton+keyresult}
\end{figure}

The contributions of this work are three-fold. First, we provide the classified performance evaluations of existing Image-based algorithms. Second, the origins are the limitations and reason of seemingly well working of the existing algorithms are  identified. Finally, a new pipeline is proposed to tackles the identified problems. The proposed pipeline not simply solves the identified problems but it also demonstrate the level of effects of them.
  % intro

\section{Classified Performance Evaluation of Image-Based VTON}

\subsection{Image-based VTON algorithms}

In this Section, we started with evaluating the 2D image based VTON algorithms using a try-on cloth and a target human image. The human representation is composed of 1) heat maps for each joints 2) silhouette of human body, and 3) face and skin pixels patches (non-cloth and human identity area). It is assumed that the target human image is pre-processed for a cloth agnostic human representation by a human pose estimation like OpenPose\cite{Cao2018OpenPoseRM} and human parsing like LIP \cite{Liang2018LookIP}. % the algorithms 
We use the same dataset collected by Han et al. used in VITON\cite{Han2017VITONAI} and CP-VTON\cite{Wang2018TowardCI} papers. Here we include the SCM based-VTON, VITON \cite{Han2017VITONAI}, and  CP-VTON \cite{Wang2018TowardCI}. We included the SCM based algorithm  for a representative of non deep learning algorithm.

For clear notation, $C_i \in R^{H \times W \times3} $ denotes the try-on input cloth image 
and $ H_t = (J_t, R_t, S_t)$ is a human representation, 
where $ J_t \in R^{H \times W \times J}$, $R_t \in R^{H \times W \times 3}$, and $ S_t \in R^{H \times W}$ are a joint heat-map, residual (e.g hands, faces, hairs) color pixel images, and body shape silhouette merged human parsing segmentations from a human image $I_i$, respectively.    

The previous algorithms are mostly composed of two stages: (1) cloth warping step that warps the try-on cloth to align with the pose and shape of the target model (called GMM in CP-VTON \cite{Wang2018TowardCI}: geometric Manipulation Module), and (2) blending step that blends the warped cloth onto the target human image (called TON in CP-VTON: Try-On Network). 

The cloth warping step is done using SCM matching and transform from try-on cloth to the current, target cloth segmentation area, and the final VTON image is blended using simple alpha blending. For VITON and CP-VTON, a brief summary is as follows. For more details, the reader is refer to the original papers.


GMM of SCM-based VTON is defined by  
$
   \bold{\theta} = f_{SCM} (C_t, C_i)
$
and
$
   C_{warped} = f_{TPS}(\theta, Ci)
$.    
GMM of VITON is defined by 
$
   (I_{r}, M_{warped}) = f_{\theta, VITON} (H_t, C_i) 
$
and the loss function for training, 
\begin{equation}
   L_{GMM}^{VITON} =   \sum_{i=0}^{5} \lambda_i || \phi (I_r) - \phi (I_i)||  + 
   ||M_{C_{warped}} -  M_{C_t}||_1 .
\end{equation}

GMM of CP-VTON, GMM is defined by 
$  
   \bold{\theta} = f_{\theta} ( f_H (H_t), f_C(C_i) )
$
and
$
   C_{warped} = f_{TPS}(\theta, Ci).
$    
with the training loss function
\begin{equation}
   L_{GMM}^{CP-VTON} =  \sum ||C_{warped}, C_t||_1
\end{equation}

% GMM
The VITON use an encoder-decoder network to generate the warped cloth mask. Then the estimated TPS parameter from SCM matching between the generated mask and input try-on cloth mask is applied to the input try-on cloth. This is because in general the generated images from an encoder-decoder network does not preserved the original cloth as well as the warped image by geometric transform. Instead, CP-VTON uses a CNN based correlation network to estimated the TPS parameter directly and use the estimated parameters for warping.   


TOM  of SCM-based VTON  is defined by 
$
   M = f_{TPS}(\theta, M_{Ci}),
$    
and 
$
   I_o = M \odot C_{warped}, + (1-M) \odot I_i 
$.
TOM of VITON is defined by 
$
   \bold{\theta} = f_{SCM} (M_{C_t}, M_{C_i})
$,
$
   C_{warped} = f_{TPS}(\theta, Ci)
$,    
$
 M = f_{TOM} ( I_{r1}, C_{warped} ) 
$, 
and 
$
   I_o = M \odot C_{warped}, + (1-M) \odot I_{r},
$
with the training loss function, 
\begin{equation}
   L_{TOM}^{VITON} = \lambda_{VGG} \sum_{i=3}^{5} \lambda_i || \phi_i(I_o) - \phi_i(I_t)||_{1}  + 
             \lambda_{warp}  || M ||_{1}  + 
             \lambda_{TV} || \nabla M||_{1}      
\end{equation}

TOM of CP-VTON is defined by 
$
 (M, I_r) = f_{TOM} ( H_t, C_{warped} )  
$
and 
$
   I_o = M \odot C_{warped} + (1-M) \odot I_r
$
with the training loss function, 
\begin{equation}
   L_{TOM}^{CP-VTON} = \lambda_{L1}  || I_o - I_t ||_1  + 
             \lambda_{VGG} \sum_{i=1}^{5} || \phi_i(I_o) - \phi_i(I_t)||_1  + 
             \lambda_{mask} || 1 - M ||_1      
\end{equation}



% TOM
In TOM, the blending step, VITON generates an alpha blending mask image, which is used when alpha-blended with warped try-on cloth and human input image. The authors of CP-VTON observed that the blending is not successful when the warped  cloth area does not align well with the target area of human image, and first generates a coarse VTON results in addition to the composition mask, and then alpha-blends the warped cloth and the coarse VTON image with the alpha-composition mask.       


Differently from other GAN applications where the main purpose is to generate an unseen, plausible image, VTON application needs to retain the original texture and shape of the input try-on cloth. Due to its higher performance and logical architecture, CP-VTON \cite{Wang2018TowardCI} has become the benchmark benchmark algorithm for the image-based VTON algorithms recently published 
% \cite{     }.  But we can not include them in this paper because the implementations are not available. However, we believe the works share the strength and problems with three algorithms examined here. 


%The previous and following in 2019 share same input image and information conditions with CP-VTON and compare the results with it.
% Add some paper summary after CP-VTON .... and explain the differences from our works.

 
\subsection{Sample Classification}

% The criteria for classification   
First of all, the criteria for classifying the experimental samples were divided according to the complexity of the try-on clothes and the target human. The target human images are classified according to the complexity of the degree of occlusion (B, OP, OB, OF), posture (P), and shape (S) of the target cloth area of the human images. For the complexity of try-on cloths, we only uses the long sleeved (L) and short sleeved (S), because the length of sleeves is the biggest component in cloth warping and the used dataset does not have full diversity in cloth  style.    

%The degree of obscuration is a factor that affects the accuracy of the object of deformation, the posture is the degree of deformation, and the complexity of the clothes means the processing complexity of the clothes themselves.
%However, it is included in the range of classification, but not included in the actual experiment is shown in parentheses. Excluded conditions are those that are not included in the test data or that the evaluation is considered to be complex in the current technology. Based on this, six cases were classified as follows.

\begin{itemize}

\item[$\bullet$] BS : no or little occlusion and posture (Short sleeved clothes)
\item[$\bullet$] BL: no or little occlusion and posture (Long sleeved clothes)
\item[$\bullet$] OP: the target cloth area in a human image Occluded Partially by her/his hair and arms
\item[$\bullet$] OB: a large part of the clothing is Occluded by her/his Bottoms.
\item[$\bullet$] OFS: the target cloth area covered by the arms (with long sleeve and short sleeve)
\item[$\bullet$] OFS: the target cloth area Occluded in Front by her/his arms (Short sleeved clothes)
\item[$\bullet$] OFL: the target cloth area Occluded in Front by her/his arms (Long sleeved clothes)

\item[$\bullet$] P:  a large Posture deformation (e.g. a large movement of the arm or twisted or lateral posture)
\item[$\bullet$] S:  a large body shape change  (over-weighted or pregnant)

\end{itemize}

The cloth shape in the VITON dataset are mostly simple in shape and texture, e.g. a T-shirt (without a collar) of  monochromatic or monotone patterns. Even though it is clearly biased dataset, it is also not easy to define the un-biased one. Although all 2023 test images are used for experiment, some categories, e.g. S, does not have only small samples. But the trends of results in each category are very clear. 


 
 
\subsection{Result} 



For evaluation of performance, we used same cloth re-try on and new cloth try-on. 
The same cloth re-try-on experiment are done for quantitative evaluation of the algorithm with the ground truth results. In the same clothing experiment, we evaluate the cloth warping step performance using IoU (Intersection over Union) and the blending step performance using average SSIM (Structured SIMilarity index).

\begin{equation}
 IoU = \frac{ \hat{M}_{C_{warped}} \cap M_{C_{GT}}}
            { \hat{M}_{C_{warped}} \cup M_{C_{GT}}},
\end{equation} 
where $\hat{M}_{C_{warped}}$ and $M_{C_{GT}}$ are the estimated warped cloth's mask and the cloth area in the target human image, respectively. 

\begin{equation}
 SSIM = \frac{ ( 2 \mu_x \mu_y + c_1 )(2 \sigma_{xy} + c_2) }
             { ( \mu^2_x + \mu^2_y +c_1) ( \sigma^2_x + \sigma^2_y + c_2)}, 
\end{equation}
where we used default parameters for radius, $c_1$, $c_2$, and $c_3$. 

The experiment results are summarized as follows:

\begin{itemize}

\item[$\bullet$] BS $\&$ BL : All three algorithms show similar pattern. They generates natural synthesized VTON images for short-sleeved cloth, but show noticeable misalignment of warped cloth sleeves with the human image for long sleeved clothes. This larger misalignment in sleeve part can be explained by the fact that the matching and deformation are mainly done globally based on the whole silhouette, but not optimized for local deformation in the sleeves. 
TOM stages of VITON and CP-VTON can hide this misalignment to some degree by blending the un-covered target cloth area. However note that TOM networks can mimics but cannot retain the original cloth texture. 

\item[$\bullet$] OP: Partial occlusion by hair and other parts causes wrong human silhouette for GMM stage, which is designed for cloth-agnostic silhouette. All three algorithms affected this strongly. 

\item[$\bullet$] OB: The bottom occlusion cause SCM GMM generate unnatural deformation for long cloths. It is because SCM algorithm does not distinguish between deformation and occlusion and use the target cloth area directly for matching. In contrast to SCM-based GMM, VITON and CP-VITON GMMs use cloth-agnostic silhouettes and the effects are no or negligible.  

\item[$\bullet$] OF: GMM cannot generate natural warped clothes for long-sleeved clothes due to the limitations in matching and deformation.  TOM of all algorithms cannot generate the composition mask so the arm area gets blurred.  

\item[$\bullet$] P: All three algorithms fails to generate the warped clothes, but Interestingly the blending stage synthesizes the cloth color and texture to some degree. Nevertheless, the synthesize texture seldom align with original textures. As described  below, this is one of inherent and critical limitations of image based VTON approach.  


\item[$\bullet$] S: In VITON dataset, the matching cloth images for the human image of large or pregnant person are also large so the effects is not serious for the same cloth re-try-on.
\end{itemize}

 
 For new cloth try-on experiments, we provide visual analysis results. The wearing of a new cloth is, in effect, the ultimate result of the application. As above, an objective evaluation is not possible with VITON dataset. In the section 3, we compare the CP-VTON result and our proposed one with Inception Score (IS),too.
 
\begin{itemize}

\item[$\bullet$] BS $\&$ BL: The clothing deformation itself shows good results similar to the result with same cloths, but there are some differences according to the algorithms in the synthesis. When switching from long to short sleeves, SCM does not restore the skin area, whereas VITON and CP-VTON can generate the skin area for the revealing bare arm. Especially due to the coarse VTON output of it, CP-VTON can generate much clear skin area.

\item[$\bullet$] OP: The performance was similar to that of the same clothes.

\item[$\bullet$] OBS $\&$ OBL: The same characteristics as the same image, that is, the SCM shows a problem that can not distinguish the occlusion and deformation.

\item[$\bullet$] OF: As in the case of the same cloth, in the case of SCM, the synthesis algorithm needs to be improved when switching from the short sleeved to the long sleeved.

\item[$\bullet$] P: As in the case of the same cloth, there was a large error in deformation and alignment of the try-on cloth.

\item[$\bullet$] S: Results shows GMM, especially of VITON and CP-VTON, can warp the new cloth to the target human body shape change, 

\end{itemize}

%In addition to the analysis of each condition in addition to the analysis of each condition, it can be seen that there are the following big features. First, it was confirmed that the shape change of clothes by GMM has an influence on the current wear clothes. The reason is that SCMM uses the area of the current costume, and deep learning methods use the body itself, but the area of the current costume is reflected in the correct answer mask used in the learning process.


\begin{figure}
\centering
%\includegraphics[height=13.5cm, scale=1]{figures/2dvton_same.png}   
\begin{tabular}{cccccccccccc}
 & \multicolumn{2}{c}{input}   &  \multicolumn{2}{c}{SCM-based}    & \multicolumn{2}{c}{VITON}    &  \multicolumn{2}{c}{CP-VTON}  & \multicolumn{2}{c}{CP-VTON+ (ours)} \\
BS &
   \includegraphics[ width=1cm, keepaspectratio]{figures/same/bsc.png} &
   \includegraphics[ width=1cm, keepaspectratio]{figures/same/bsh.png} &
   \includegraphics[ width=1cm, keepaspectratio]{figures/same/bsscmc.png} &
   \includegraphics[ width=1cm, keepaspectratio]{figures/same/bsscmh.png} &
   \includegraphics[ width=1cm, keepaspectratio]{figures/same/bsvitonc.png} &
   \includegraphics[ width=1cm, keepaspectratio]{figures/same/bsvitonh.png} &
   \includegraphics[ width=1cm, keepaspectratio]{figures/same/bsvtonc.png} &
   \includegraphics[ width=1cm, keepaspectratio]{figures/same/bsvtonh.png} &
   \includegraphics[ width=1cm, keepaspectratio]{figures/same/bsvton+c.png} &
   \includegraphics[ width=1cm, keepaspectratio]{figures/same/bsvton+h.png}  \\
 &&& 
   0.955 & 0.882 &
   0.708 & 0.665 & 
   0.800 & 0.690 &
   0.933 & 0.785\\   
   
BL &
   \includegraphics[ width=1cm, keepaspectratio]{figures/same/blc.png} &
   \includegraphics[ width=1cm, keepaspectratio]{figures/same/blh.png} &
   \includegraphics[ width=1cm, keepaspectratio]{figures/same/blscmc.png} &
   \includegraphics[ width=1cm, keepaspectratio]{figures/same/blscmh.png} &
   \includegraphics[ width=1cm, keepaspectratio]{figures/same/blvitonc.png} &
   \includegraphics[ width=1cm, keepaspectratio]{figures/same/blvitonh.png} &
   \includegraphics[ width=1cm, keepaspectratio]{figures/same/blvtonc.png} &
   \includegraphics[ width=1cm, keepaspectratio]{figures/same/blvtonh.png} &
   \includegraphics[ width=1cm, keepaspectratio]{figures/same/blvton+c.png} &
   \includegraphics[ width=1cm, keepaspectratio]{figures/same/blvton+h.png}  \\
 &&& 
     0.784 & 0.872 &
     0.652 & 0.724 &
     0.830 & 0.881 &
     0.862 & 0.878\\

OP &
   \includegraphics[ width=1cm, keepaspectratio]{figures/same/opc.png} &
   \includegraphics[ width=1cm, keepaspectratio]{figures/same/oph.png} &
   \includegraphics[ width=1cm, keepaspectratio]{figures/same/opscmc.png} &
   \includegraphics[ width=1cm, keepaspectratio]{figures/same/opscmh.png} &
   \includegraphics[ width=1cm, keepaspectratio]{figures/same/opvitonc.png} &
   \includegraphics[ width=1cm, keepaspectratio]{figures/same/opvitonh.png} &
   \includegraphics[ width=1cm, keepaspectratio]{figures/same/opvtonc.png} &
   \includegraphics[ width=1cm, keepaspectratio]{figures/same/opvtonh.png} &
   \includegraphics[ width=1cm, keepaspectratio]{figures/same/opvton+c.png} &
   \includegraphics[ width=1cm, keepaspectratio]{figures/same/opvton+h.png}  \\
 &&&  
     0.872 &  0.862	&
     0.747 &  0.708	&
     0.834 &  0.824 &	  
     0.877 &  0.863\\
     
     

OBS &
   \includegraphics[ width=1cm, keepaspectratio]{figures/same/obsc.png} &
   \includegraphics[ width=1cm, keepaspectratio]{figures/same/obsh.png} &
   \includegraphics[ width=1cm, keepaspectratio]{figures/same/obsscmc.png} &
   \includegraphics[ width=1cm, keepaspectratio]{figures/same/obsscmh.png} &
   \includegraphics[ width=1cm, keepaspectratio]{figures/same/obsvitonc.png} &
   \includegraphics[ width=1cm, keepaspectratio]{figures/same/obsvitonh.png} &
   \includegraphics[ width=1cm, keepaspectratio]{figures/same/obsvtonc.png} &
   \includegraphics[ width=1cm, keepaspectratio]{figures/same/obsvtonh.png} &
   \includegraphics[ width=1cm, keepaspectratio]{figures/same/obsvton+c.png} &
   \includegraphics[ width=1cm, keepaspectratio]{figures/same/obsvton+h.png}  \\
 &&&  
      0.954  &       0.858	&  
       0.731 &        0.696	 & 
       0.525 &         0.732 &	  
           0.933 &       0.804\\
OBL &
   \includegraphics[ width=1cm, keepaspectratio]{figures/same/oblc.png} &
   \includegraphics[ width=1cm, keepaspectratio]{figures/same/oblh.png} &
   \includegraphics[ width=1cm, keepaspectratio]{figures/same/oblscmc.png} &
   \includegraphics[ width=1cm, keepaspectratio]{figures/same/oblscmh.png} &
   \includegraphics[ width=1cm, keepaspectratio]{figures/same/oblvitonc.png} &
   \includegraphics[ width=1cm, keepaspectratio]{figures/same/oblvitonh.png} &
   \includegraphics[ width=1cm, keepaspectratio]{figures/same/oblvtonc.png} &
   \includegraphics[ width=1cm, keepaspectratio]{figures/same/oblvtonh.png} &
   \includegraphics[ width=1cm, keepaspectratio]{figures/same/oblvton+c.png} &
   \includegraphics[ width=1cm, keepaspectratio]{figures/same/oblvton+h.png}  \\
&&&  
      0.714 &  0.934 &	  
      0.428 &  0.789 &	  
      0.790 &  0.730 &	  
      0.816 &  0.897\\

OF &
   \includegraphics[ width=1cm, keepaspectratio]{figures/same/ofc.png} &
   \includegraphics[ width=1cm, keepaspectratio]{figures/same/ofh.png} &
   \includegraphics[ width=1cm, keepaspectratio]{figures/same/ofscmc.png} &
   \includegraphics[ width=1cm, keepaspectratio]{figures/same/ofscmh.png} &
   \includegraphics[ width=1cm, keepaspectratio]{figures/same/ofvitonc.png} &
   \includegraphics[ width=1cm, keepaspectratio]{figures/same/ofvitonh.png} &
   \includegraphics[ width=1cm, keepaspectratio]{figures/same/ofvtonc.png} &
   \includegraphics[ width=1cm, keepaspectratio]{figures/same/ofvtonh.png} &
   \includegraphics[ width=1cm, keepaspectratio]{figures/same/ofvton+c.png} &
   \includegraphics[ width=1cm, keepaspectratio]{figures/same/ofvton+h.png}  \\
&&&  
     0.855 & 0.866 &	  
     0.727 & 0.660 &	  
     0.841 & 0.865 &	  
     0.847 & 0.874\\
P &
   \includegraphics[ width=1cm, keepaspectratio]{figures/same/pc.png} &
   \includegraphics[ width=1cm, keepaspectratio]{figures/same/ph.png} &
   \includegraphics[ width=1cm, keepaspectratio]{figures/same/pscmc.png} &
   \includegraphics[ width=1cm, keepaspectratio]{figures/same/pscmh.png} &
   \includegraphics[ width=1cm, keepaspectratio]{figures/same/pvitonc.png} &
   \includegraphics[ width=1cm, keepaspectratio]{figures/same/pvitonh.png} &
   \includegraphics[ width=1cm, keepaspectratio]{figures/same/pvtonc.png} &
   \includegraphics[ width=1cm, keepaspectratio]{figures/same/pvtonh.png} &
   \includegraphics[ width=1cm, keepaspectratio]{figures/same/pvton+c.png} &
   \includegraphics[ width=1cm, keepaspectratio]{figures/same/pvton+h.png}  \\
&&&  
     0.762 & 0.619 &	  
     0.756 & 0.553 &	  
     0.835 & 0.666 &	  
     0.841 & 0.663\\
S &
   \includegraphics[ width=1cm, keepaspectratio]{figures/same/sc.png} &
   \includegraphics[ width=1cm, keepaspectratio]{figures/same/sh.png} &
   \includegraphics[ width=1cm, keepaspectratio]{figures/same/sscmc.png} &
   \includegraphics[ width=1cm, keepaspectratio]{figures/same/sscmh.png} &
   \includegraphics[ width=1cm, keepaspectratio]{figures/same/svitonc.png} &
   \includegraphics[ width=1cm, keepaspectratio]{figures/same/svitonh.png} &
   \includegraphics[ width=1cm, keepaspectratio]{figures/same/svtonc.png} &
   \includegraphics[ width=1cm, keepaspectratio]{figures/same/svtonh.png} &
   \includegraphics[ width=1cm, keepaspectratio]{figures/same/svton+c.png} &
   \includegraphics[ width=1cm, keepaspectratio]{figures/same/svton+h.png}  \\
&&&  
      0.954 & 0.715	&  
      0.759 & 0.553	&  
      0.884 & 0.637 &	  
      0.888 & 0.652\\
\end{tabular}

\caption{Classified Evaluation of Image-based VTONs (same cloth re-try-on): left to right, input pairs, SCM-based results, VITON results, CP-VTON results, CP-VTON+ (ours) results.  The numbers below cloth and human images are IoU and SSIM, respectively. }
\label{fig:2dvton_same}
\end{figure}



\begin{figure}
\centering
%\includegraphics[height=13.5cm, scale=1]{figures/2dvton_diff.png}  %% TODO  

\begin{tabular}{cccccccccccc}

  &  \multicolumn{2}{c}{input}  & \multicolumn{2}{c}{SCM-based}   & \multicolumn{2}{c}{VITON}    &  \multicolumn{2}{c}{CP-VTON}  & \multicolumn{2}{c}{CP-VTON+ (ours)} \\
BS &
   \includegraphics[ width=1cm, keepaspectratio]{figures/new/bsc.png} &
   \includegraphics[ width=1cm, keepaspectratio]{figures/new/bsh.png} &
   \includegraphics[ width=1cm, keepaspectratio]{figures/new/bsscmc.png} &
   \includegraphics[ width=1cm, keepaspectratio]{figures/new/bsscmh.png} &
   \includegraphics[ width=1cm, keepaspectratio]{figures/new/bsvitonc.png} &
   \includegraphics[ width=1cm, keepaspectratio]{figures/new/bsvitonh.png} &
   \includegraphics[ width=1cm, keepaspectratio]{figures/new/bsvtonc.png} &
   \includegraphics[ width=1cm, keepaspectratio]{figures/new/bsvtonh.png} &
   \includegraphics[ width=1cm, keepaspectratio]{figures/new/bsvton+c.png} &
   \includegraphics[ width=1cm, keepaspectratio]{figures/new/bsvton+h.png}  \\
   
BL &
   \includegraphics[ width=1cm, keepaspectratio]{figures/new/blc.png} &
   \includegraphics[ width=1cm, keepaspectratio]{figures/new/blh.png} &
   \includegraphics[ width=1cm, keepaspectratio]{figures/new/blscmc.png} &
   \includegraphics[ width=1cm, keepaspectratio]{figures/new/blscmh.png} &
   \includegraphics[ width=1cm, keepaspectratio]{figures/new/blvitonc.png} &
   \includegraphics[ width=1cm, keepaspectratio]{figures/new/blvitonh.png} &
   \includegraphics[ width=1cm, keepaspectratio]{figures/new/blvtonc.png} &
   \includegraphics[ width=1cm, keepaspectratio]{figures/new/blvtonh.png} &
   \includegraphics[ width=1cm, keepaspectratio]{figures/new/blvton+c.png} &
   \includegraphics[ width=1cm, keepaspectratio]{figures/new/blvton+h.png}  \\

OP &
   \includegraphics[ width=1cm, keepaspectratio]{figures/new/opc.png} &
   \includegraphics[ width=1cm, keepaspectratio]{figures/new/oph.png} &
   \includegraphics[ width=1cm, keepaspectratio]{figures/new/opscmc.png} &
   \includegraphics[ width=1cm, keepaspectratio]{figures/new/opscmh.png} &
   \includegraphics[ width=1cm, keepaspectratio]{figures/new/opvitonc.png} &
   \includegraphics[ width=1cm, keepaspectratio]{figures/new/opvitonh.png} &
   \includegraphics[ width=1cm, keepaspectratio]{figures/new/opvtonc.png} &
   \includegraphics[ width=1cm, keepaspectratio]{figures/new/opvtonh.png} &
   \includegraphics[ width=1cm, keepaspectratio]{figures/new/opvton+c.png} &
   \includegraphics[ width=1cm, keepaspectratio]{figures/new/opvton+h.png}  \\

OBS &
   \includegraphics[ width=1cm, keepaspectratio]{figures/new/obsc.png} &
   \includegraphics[ width=1cm, keepaspectratio]{figures/new/obsh.png} &
   \includegraphics[ width=1cm, keepaspectratio]{figures/new/obsscmc.png} &
   \includegraphics[ width=1cm, keepaspectratio]{figures/new/obsscmh.png} &
   \includegraphics[ width=1cm, keepaspectratio]{figures/new/obsvitonc.png} &
   \includegraphics[ width=1cm, keepaspectratio]{figures/new/obsvitonh.png} &
   \includegraphics[ width=1cm, keepaspectratio]{figures/new/obsvtonc.png} &
   \includegraphics[ width=1cm, keepaspectratio]{figures/new/obsvtonh.png} &
   \includegraphics[ width=1cm, keepaspectratio]{figures/new/obsvton+c.png} &
   \includegraphics[ width=1cm, keepaspectratio]{figures/new/obsvton+h.png}  \\

OBL &
   \includegraphics[ width=1cm, keepaspectratio]{figures/new/oblc.png} &
   \includegraphics[ width=1cm, keepaspectratio]{figures/new/oblh.png} &
   \includegraphics[ width=1cm, keepaspectratio]{figures/new/oblscmc.png} &
   \includegraphics[ width=1cm, keepaspectratio]{figures/new/oblscmh.png} &
   \includegraphics[ width=1cm, keepaspectratio]{figures/new/oblvitonc.png} &
   \includegraphics[ width=1cm, keepaspectratio]{figures/new/oblvitonh.png} &
   \includegraphics[ width=1cm, keepaspectratio]{figures/new/oblvtonc.png} &
   \includegraphics[ width=1cm, keepaspectratio]{figures/new/oblvtonh.png} &
   \includegraphics[ width=1cm, keepaspectratio]{figures/new/oblvton+c.png} &
   \includegraphics[ width=1cm, keepaspectratio]{figures/new/oblvton+h.png}  \\

OF &
   \includegraphics[ width=1cm, keepaspectratio]{figures/new/ofc.png} &
   \includegraphics[ width=1cm, keepaspectratio]{figures/new/ofh.png} &
   \includegraphics[ width=1cm, keepaspectratio]{figures/new/ofscmc.png} &
   \includegraphics[ width=1cm, keepaspectratio]{figures/new/ofscmh.png} &
   \includegraphics[ width=1cm, keepaspectratio]{figures/new/ofvitonc.png} &
   \includegraphics[ width=1cm, keepaspectratio]{figures/new/ofvitonh.png} &
   \includegraphics[ width=1cm, keepaspectratio]{figures/new/ofvtonc.png} &
   \includegraphics[ width=1cm, keepaspectratio]{figures/new/ofvtonh.png} &
   \includegraphics[ width=1cm, keepaspectratio]{figures/new/ofvton+c.png} &
   \includegraphics[ width=1cm, keepaspectratio]{figures/new/ofvton+h.png}  \\

P &
   \includegraphics[ width=1cm, keepaspectratio]{figures/new/pc.png} &
   \includegraphics[ width=1cm, keepaspectratio]{figures/new/ph.png} &
   \includegraphics[ width=1cm, keepaspectratio]{figures/new/pscmc.png} &
   \includegraphics[ width=1cm, keepaspectratio]{figures/new/pscmh.png} &
   \includegraphics[ width=1cm, keepaspectratio]{figures/new/pvitonc.png} &
   \includegraphics[ width=1cm, keepaspectratio]{figures/new/pvitonh.png} &
   \includegraphics[ width=1cm, keepaspectratio]{figures/new/pvtonc.png} &
   \includegraphics[ width=1cm, keepaspectratio]{figures/new/pvtonh.png} &
   \includegraphics[ width=1cm, keepaspectratio]{figures/new/pvton+c.png} &
   \includegraphics[ width=1cm, keepaspectratio]{figures/new/pvton+h.png}  \\

S &
   \includegraphics[ width=1cm, keepaspectratio]{figures/new/sc.png} &
   \includegraphics[ width=1cm, keepaspectratio]{figures/new/sh.png} &
   \includegraphics[ width=1cm, keepaspectratio]{figures/new/sscmc.png} &
   \includegraphics[ width=1cm, keepaspectratio]{figures/new/sscmh.png} &
   \includegraphics[ width=1cm, keepaspectratio]{figures/new/svitonc.png} &
   \includegraphics[ width=1cm, keepaspectratio]{figures/new/svitonh.png} &
   \includegraphics[ width=1cm, keepaspectratio]{figures/new/svtonc.png} &
   \includegraphics[ width=1cm, keepaspectratio]{figures/new/svtonh.png} &
   \includegraphics[ width=1cm, keepaspectratio]{figures/new/svton+c.png} &
   \includegraphics[ width=1cm, keepaspectratio]{figures/new/svton+h.png}  \\

\end{tabular}

\caption{Classified Evaluation of Image-based VTONs (new cloth try-on): left to right, input pairs, SCM-based results, VITON results, CP-VTON results, CP-VTON+ (ours) results.}
\label{fig:2dvton_diff}
\end{figure}

 
\subsection{Discussion}

% comparison between algorithm
Among the three algorithms used, CP-VTON shows the best performance in both same cloth re-try-on and new cloth try-on. However there are some notable observations. 

As for the warping step, interestingly, SCM-based GMM generates the best results in the alignment and deformation of the try-on clothes to the target human for the same cloth. This somewhat conflicting against \cite{Wang2018TowardCI} can be explained as follows. In \cite{Wang2018TowardCI}, they only considered new cloth try-on cases.  Note that SCM can use the ground truth target in same cloth re-try-on so that it can find better matching points. However, for the new cloth cases, when the shapes of new cloth and current clothes has big difference, the results get worse than the other methods.
  
And in general, CP-VTON GMM using CNN Geometry matching algorithm shows better results than VITON GMM. Since both CP-VTON and VITON use TPS algorithm, the difference comes from TPS parameter estimation step. We can conclude the CNN-based matching works better that of VITON. Nevertheless the matching is not very reliable and the warped cloth show strong distortion in texture, e.g. pattern and logos. 

Examination the sample they argued and tried to use cloth-agnostic human representation, the current VITON dataset target try-on area is dependent upon current cloth shape. Especially, the neck area pixels are labeled as background and some body areas are occluded by hairs or accessories (Figure \ref{fig:cpvtonissues}), which affects in cloth warping and blending. 

 
GMM module using Spatial Transform Network\cite{JaderbergSZK15} with TPS (Thin Plate Spline)\cite{Bookstein1989PrincipalWT} deformation cannot handle strong 3D deformation due to the target pose and also generates artifacts because of the person representation inputs. For example, hands-up and folded arms.  Note that many errors in the warping stage are often hidden in the blending stage when the target clothes are single-colored, which can be expected in practical conditions 
%(Fig. 3 (d)).

      
As for the blending stage, CP-VTON performs best among three algorithms. SCM generates clear texture but a noticeable cloth boundaries due to the binary mask used for blending.  CP-VTON generated much clearer cloth texture thanks to its composition mask and synthesized VTON result.

% 2.
%Secondly, 
All the other parts but the target cloth area, e.g. faces, hair, and arms, bottom-clothes have to be retained in blending stage. But other parts except face and hair are missing in CP-VTON\cite{Wang2018TowardCI} human representation and generated at blending stage, which is all right for general synthesis application but not desirable in VTON application (Figure \ref{fig:cpvtonissues}). 


% 3.
The texture is often not vivid, which is due to the unclear composition mask. Examining the original loss function of TOM network, the term for the composition alpha mask are poorly formulated as simple regularization loss $|| 1 - M_{o} ||_1$.   

%
%\begin{equation}
%L = \lambda_{L1} || I_0-I_{GT} ||_1 +  \lambda_{VGG} L_{VGG}+ \lambda_3 || 1 - M_{o} ||_1        
%\end{equation} 


%4. 
Fourthly, since no label in the area of warped cloth is the same color as background, white colored clothes are confused and improperly processed in the blending stage (Fig. 3 (c))


%All the used image-based algorithms shows similar trends in each input
%Even though the success and failure cases are presented and compared with other algorithms' results, the failure case analysis is not enough for understanding the origin of failure cases and therefore difficult to find the solution for them. A classified evaluation would be better for this understanding. Here we summarized the classified results from our another study. We classify input try-on cloth and target human images according to the posture and body type of the person, the degree of occlusion of the clothes, and the characteristics of the clothes. Quality is compared in IoU for the warping step and in SSIM for the final blending step for same cloth re-try-on cases. We also tested for the new cloth try-on cases but did not include here for limitation of spaces, and the same cloth cases are enough to explain the tendencies of the performances. Though in general CP-VTON generates the best quality image, the relative comparison is not the main purpose of the analysis. Please refer to Figure \ref{fig:classified2DVTONresult} for detailed results.

\begin{figure}
\centering
\includegraphics[height=6.5cm, scale=1]{figures/cpvtonissues.png}   % TODO
\caption{The issues in CP-VTON}
\label{fig:cpvtonissues}
\end{figure}



%Especially note that the warped cloth are often too much different for desired shape. It is originated two facts. First the 3D deformation that any 2D deformation including non-rigid transform such as TPS is quite limited, especially any 2D deformation cannot handle when the two area in the original image are overlapped in the destination images. There for when the arms of long sleeved cloth occlude the main body, 2D warping cannot approximate the 3D deformation properly. Second, the deformation needs corresponding points  between the source nd target image. The cloth are extremely difficult object to find the corresponding points. The STN (spatial transform network)\cite{JaderbergSZK15} and SCM (shape context matching)\cite{BelongieMP02} cannot find the corresponding points when the target cloth and original cloth has different shapes. In conclusion, the 2D image based algorithm has serious limitation in the range of applications. It can apply to the mild posed target human only and simple short sleeved cloth, mainly because the inherent limitation of 2D deformation method including non-rigid ones, and the poor performance of matching algorithm.  To overcome this limitation, we consider to model the try-on cloth into 3D model and apply the 3D deformation
  % Classified analysis


\section{CV-VTON+} \label{section:cpvton+}

\subsection{Overview} 

Fig. \ref{fig:piepline} illustrates the new VTON pipeline that tackle the problem mentioned above, based on the pipeline structure of CV-VTON, hence named CP-VTON+. The modified components are highlighted for comparison, described as follows.

\subsubsection{GMM Modification}

\begin{itemize}

\item[$\bullet$]  Correction on the cloth agnostic human representation:
It is crucial to obtain the complete target body area for the try-on cloth and the target area should be the current-cloth agnostic. original VITON dataset does not satisfy this requirement.  
% BG2SKIN
To be independent upon the current cloth the human wear, a new label ‘skin’ is added to the label set,  and the pixels wrongly labelled as 'background' in VITON dataset are replaced by 'skin' considering the joint locations. The skin-labelled area now is included in the silhouette in the human representation (Figure  ~\ref{fig:cpvtonissues}).
% HAIR-OCCLUSION
To recover the hair occlusion, first the hair occlusion area are identified as the intersection of convex contour of the upper cloth and the hair-labelled area, and the intersection are re-labelled as upper cloth.   


\item[$\bullet$] GIC Network modification: 
CP-VTON GMM network is built on CNN geometric matching \cite{rocco2017convolutional}, and trained in self-supervised manner by the L1 loss between the warped cloth images and the current human cloth. It is quite interesting that CNN geometric matching uses a pair of color images, but CP-VTON GMM uses a human representation, e.g. silhouette and joint heatmap with the try-on cloth.  The human representation only provide the shape and structure information and the try-on cloth provides both texture and shape information. Furthermore the texture information is not useful for finding the corresponding locations. The experiments above revealed the warped cloth often distorted severely. We tried to understand the origin of this distortion, but failed. From this two observation, the new GMM is design to use mask of try-on cloth, i.e. taking into account the only shape information of the try-on cloth, and regularize the transform parameter in some restricted range by adding regularization loss to the training network. The grid warping regularization is defined not to have too much different warping from the before and next grid gap in equation ~\ref{eq:gridloss}.

 
%, which again uses the spatial transform networks by Zimmerman et al.'s \cite{jaderberg2015spatial}. Although warping network can work in both instance-level and category-level, our experiments  reveals the warped cloth often distorted severely.


%Our finding is that the warping network try to matches cloth images with human representation of silhouette and joint heatmap. The latter does not have no texture features but the shape information. From the failure case examinations, we found that the warping in the failure case is often too extreme, so that we try to restrict the warping parameters. Instead of single step TPS warping, we designed 2 stage warping network, in the first stage it warped roughly with Affine transform, and in the second stage it warped delicately with TPS transform. Not to have un-realistic warping, we added warping regularization loss. In the second stage, the sizes of cloth and target area in the human representation are in the same scale, so that the matching network can find the matching better. 


\begin{equation}
 L  = c_1 \cdot L1(I_c, I_{c,GT}) + c_2 \cdot  L_{reg}  
\end{equation}

\begin{equation}\label{eq:gridloss}
\begin{aligned}
 L_{reg} (G_x, G_y) = \sum_x \sum_y | G_x(x+1, y) - G_x(x, y) | - | G_x(x, y) - G_x(x-1, y) | \\
 + | G_y(x, y+1) - G_y(x, y) | - | G_y(x, y) - G_y(x, y-1) |
\end{aligned}
\end{equation}


\end{itemize}

\subsubsection{TOM Modification}


\begin{itemize}

\item[$\bullet$] Retaining the remaining area except the target cloth  :
In order to retain the other human components except the target cloth area, e.g. bottom clothes and legs, the all other area except the target cloth area are added for the human representation input of TOM, along with face and hair (Figure ~\ref{fig:cpvtonissues})

\item[$\bullet$] Training Composition alpha-map:  
In the mask loss term in TOM loss function, we replaced the Composition Mask with supervised ground truth mask for a strong alpha mask.

\begin{equation}
L=c_1 |I_0-I_{GT} |+  c_2 L_{VGG}+c_1 |M_{GT}-M_0 |       
\end{equation}

Also, we added the binary mask of warped cloth to TOM network input so that TOM can clearly differentiate the target cloth area regardless of cloth color.  


\end{itemize}


\begin{figure}
\centering
\includegraphics[height=6.5cm, scale=1]{figures/cpvton+pipeline.png}   
\caption{Full pipeline Comparison: CP-VTON and CP-VTON+}
\label{fig:piepline}
\end{figure}

%\begin{figure}
%\centering
%\includegraphics[height=6.5cm, scale=1]{figures/cpvton+pipeline.png}   
%\caption{2 stage GMM design with regularization loss}
%\label{fig:piepline}
%\end{figure}

  % CPVTON+ 

\section{Experiment and Results} 

\subsection{Implementation details} 

We used the same dataset used in CP-VTON, collected first for VITON. We used IoU and SSIM performance metrics for the same cloth retry-on cases for GMM and TOM. For final output quality measures, we used SSIM, Inception Score (IS)\cite{salimans2016improved} and Learned Perceptual Image Patch Similarity (LPIPS) metric \cite{zhang2018unreasonable} for different cloth try-on. The subjective qualities can be examined in Fig. 5/7.  Special comments are required for the IoU values, where CP-VTON (0.78) is slightly higher than CP-VTON+ (0.75). The un-expected results originated due to CP-VTON generating as in the current cloth shape. However, similar clothes are not always applicable, furthermore, it generates wrong shaped results for different clothes. Fig. 6 illustrates this with two typical example, plugging and normal tops


\subsection{Comparative Results}

\begin{figure}
\centering
%\includegraphics[height=6.5cm, scale=1]{figures/vton_result1.png} 
\includegraphics[height=6.5cm, scale=1]{figures/cpvton+compare.png} 
\caption{VTON results: We have to include 1) GMM and TOM results for demonstrating all the improvement one by one: }
\label{fig:vtonresults}
\end{figure}

\subsection{Ablation Study}

Figure ~\ref{fig:ablation} we highlight the impact of the identified problems and improvement of the proposed method step-by-step through the ablation study of CP-VTON+. The first and second columns are target humans and try-on clothes, respectively. The third column is vanilla CVP-VTON results. The fourth column is when unchanged clothes and body parts are added to the reserved region inputs of TOM, retaining the original pants texture. The fifth column is when the mask loss function of TOM is updated with the target cloth area, making the texture and color of cloth sharp and vivid. Finally the sixth and last column is when the body masks are updated, replacing the skin area wrongly labelled as background and hair are removed from the reserved region input of GMM, making GMM can better cloth-and-hair-agnostic human representation.  

WHEN WE GET the GMM improvement, Where we can add this? may be the first step?

\begin{figure}
\centering
\includegraphics[height=6.5cm, scale=1]{figures/ablation.png} 
\caption{Ablation study of CP-VTON+. From left to right column. Target human, try-on cloth, CP-VTON, w. human representation, warped cloth mask and mask loss function updated, and CP-VTON+
}
\label{fig:ablation}
\end{figure}


Final, i.e., TOM results are evaluated with non-reference methods, LPIPS, IS and SSIM. Our proposed method, CP-VTON+ outperforms CP-VTON in LPIPS with 0.1263 against 0.1397, in SSIM with 0.8076 over 0.7798, and in IS with 2.76 over 2.7417. The subjective evaluation shows significant visual improvements, especially in cloth textures such as logos and patterns (Fig. 7). We added the test results for all test cases for comparison between CP-VTON and CP-VTON+ in supplementary materials, together with the categorized comparison of SCM-based VTON, VITON and CP-VTON.


\subsection{Known further issues}

Even though our modification  improve the VTON results a lot, and showing highly natural results, note that the dataset has limited samples for difficult cases, like the long sleeved, complicated shaped, or textured cloth and large posture target human. We amplify the key problems identified not try to list all the small problems. 

As Figure ~\ref{fig:gmmfailure} shows two typical failure cases due to the cloth warping. First row shows when the arms heavily covers the body area. The warped cloth does not match to the human body and TON failed in hiding the warping error. It is due to the limitation of STN (Spatial Transform Network). STN is originally developed for invert the (augmented) input images for different camera views and camera distortion. Non-rigid transforms, including TPS algorithms, cannot handle the strong 3D deformations of cloth.  Also the 3D poses induce self-occlusions. The TON network should recognized the cloth area and skin areas, like naked arms. One practical short-term solution would be to restrict the pose of target human image from the customer. And the long-term solution would be developed an 3D cloth deformation techniques for the GMM step, which is under studied by the authors. 

The second rows shows the another problems. Even without strong 3D posture of the target human, the warped cloth often shows un-realistic results. The accuracy for matching and warping of STN is not fully studied for VTON applications. 

%%%%%%%%
WE need to say something.
%%%%
    
The output image quality of all image-based VTON algorithms including ours depend upon the quality of input human representation, i.e., estimated joint locations and parsed human segmentation. The poses of target humans are usually (or forced) rather simple so that the state-of-the-art pose estimation algorithms can provide fairly accurate positions. However the quality of parsed human are not always good enough, especially when the target human wear complicated clothes. Also one can restrict the complexity of human images in pose and cloth style, but still the accuracy at the segmentation boundary sometimes affects the blended image results as shown in third row case in Fig ~\ref{fig:ablation}, where the pixels of the current top cloth, which is mislabelled as bottom cloth, remained in the blending result. There fore high quality human parsing algorithm especially around boundaries are required.   

\begin{figure}
\centering
\includegraphics[height=6.5cm, scale=1]{figures/gmmfailure.png} 
\caption{Cloth Warping Failure of CP-VTON+. From left to right column. Target human, try-on cloth, CP-VTON, and CP-VTON+  (Why not same first and second ??)
}
\label{fig:gmmfailure}
\end{figure}
   % Experiments


\section{Conclusions}

With no need in 3-D information, image-based approach is considered an easier and practical virtual-try-on method. Through a group of paper works, a pair of a try-on cloth and a target human image as input data has become a de-facto standards setting for image-based VTON system.   
However, the previous work focuses on demonstrating the success of their methods and neural networks, but does not examine the limitations and their successfully operating range for the input cloth and human image.  

Even though the neural network systems are criticized for its black-box system properties, at least the real design of network structure, input data, and training mechanism are based upon the understanding on the cause-and-result relations. We demonstrated the examination of results based on the classified inputs conditions is crucial in understanding and improving the methods and systems. The experimental results show that the state-of-the-art methods works fairly well the cases with mono-colored short sleeved cloths and a up-front posed human, but not the cases with a rich-textured and long sleeved cloth or a diverse posed human.    

Also, in multi-stage pipeline system, the examination at intermediate results is also crucial in understanding the limitations and the working ranges. The state-of-the-art image-based VTON systems are constructed as a 2-step pipeline. Our examination of the intermediate warped clothes together with the final VTON results revealed profound problems in the-state-of-the-art methods and dataset pre-processing, e.g. erroneous cloth-agnostic human representation due to wrong labelling of the chest area, the in-ability of the non-rigid transform for 3D posed cloths, the failure in estimating transform parameters with human representation images and cloth images.    

This classified examination and observation helps us to design a modification of warping network, a better input processing and new training methods, and the proposed system outperforms the state-of-the-art methods quantitatively: in Intersection-Over-Union (over 10 percent) and Structural Similarity Index (over 7 percent) for the same cloth re-try-on, and Inception Score and visual observation (over 4 percent) for new cloth try-on test.     

However, the authors think image-based methods using 2-D image space have inherent limitations for covering diversely posed target human cases. Therefore, for short-term workaround, it is recommended to restrict the pose of target human image, but in the long-term solution an approaches using a 3-D mesh or voxel-based is to be required. The authors are studying model based cloth reconstruction method for this approach.

     
\clearpage
% ---- Bibliography ----
%
% BibTeX users should specify bibliography style 'splncs04'.
% References will then be sorted and formatted in the correct style.
%
\bibliographystyle{splncs04}
\bibliography{cpvton+bib}



\end{document}
